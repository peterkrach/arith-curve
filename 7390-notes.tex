\documentclass{article}

\usepackage{amsmath,amssymb,amsthm,fullpage,hyperref,mathrsfs,stmaryrd}
%\usepackage{cleveref}
\usepackage[all]{xy}
\DeclareMathOperator{\Div}{Div}
\DeclareMathOperator{\h}{H}
\DeclareMathOperator{\jac}{Jac}
\DeclareMathOperator{\ord}{ord}
\DeclareMathOperator{\pic}{Pic}
\DeclareMathOperator{\spec}{Spec}

\newtheorem{theorem}[subsection]{Theorem}
\newtheorem{definition}[subsection]{Definition}
\theoremstyle{definition}
\newtheorem{example}[subsection]{Example}
\newtheorem{question}[subsection]{Question}
\newtheorem{remark}[subsection]{Remark}

\title{Arithmetic of curves}
\author{Daniel Miller\thanks{notes to a course taught by David Zywina at Cornell}}
\date{fall 2013}

\begin{document}
\maketitle

% perhaps include illustrations (graphs) of curves?










\section{Plane curves}

Fir a non-constant polynomial $f(x,y)\in\mathbb{Q}[x,y]$. Assume $f$ is 
geometrically irreducible (i.e. irreducible in $\overline{\mathbb{Q}}[x,y]$). 
We can define the curve $C$ over $\mathbb{Q}$ determined by the equation 
$f(x,y)=0$. For now, we will think of $C$ in terms of its functor of points. 
To any $\mathbb{Q}$-algebra $A$, we have $C(A)=\{(a,b)\in A^2:f(a,b)=0\}$. 
For those of you who know about schemes, 
$C=\spec\left(\mathbb{Q}[x,y]/f\right)$. Some big questions are:
\begin{enumerate}
  \item Is $C(\mathbb{Q})=\varnothing$?
  \item Is $C(\mathbb{Q})$ finite?
  \item Can we compute $C(\mathbb{Q})$?
\end{enumerate}
None of these questions are known in full generality. 

\begin{example}
Let $f=x^2+y^2-1$, i.e. $C$ is the circle. It is well-known that 
\[
  C(\mathbb{Q})=\left\{\left(\frac{1-t^2}{1+t^2},\frac{2t}{1+t^2}\right):t\in\mathbb{Q}\right\}\cup \{(-1,0)\}
\]
This can be proved in the usual manner by choosing the point 
$(-1,0)$ in $C$, and then drawing lines with rational slopes through 
$(-1,0)$. As a Riemann surface, $C(\mathbb{C})$ is a sphere minus 
two points. 
\end{example}

\begin{example}
Let $f=x^2+y^2+1$. Then $C(\mathbb{R})=\varnothing$, hence 
$C(\mathbb{Q})=\varnothing$. But as a Riemann surface, $C(\mathbb{C})$ is 
the same sphere minus two points. Thus the geometry of $C$ does not 
necessarily determine $C(\mathbb{Q})$. 
\end{example}

\begin{example}
Let $f=x^4+y^4-1$. Then it is a theorem of Fermat that 
$C(\mathbb{Q})=\{(\pm 1,0),(0,\pm 1)\}$.
\end{example}

\begin{example}[Stoll]
Consider 
\begin{align*}
  C : y^2 = 82342800 x^6 &- 470135160 x^5 + 52485681 x^4 + 2396040466 x^3 \\
    &+ 567207969 x^2 - 985905640 x + 247747600
\end{align*}
This is a curve of genus 2, i.e. $C(\mathbb{C})$ is a punctured two-hole 
torus. It turns out that $\# C(\mathbb{Q})\geqslant 642$ \cite[\S 6]{st09}. 
However, by a theorem of Faltings, $\# C(\mathbb{Q})$ is finite. It is 
currently the largest known number of rational points of a genus two curve 
over $\mathbb{Q}$. If we take $C:y^2=f(x)$ with $f\in\mathbb{Q}[x]$ a 
``random'' sextic polynomial, then the expectation is that 
$C(\mathbb{Q})=\varnothing$. 
\end{example}

\begin{question}
Does there exist a number $B_g$ such that if a curve $C/\mathbb{Q}$ has 
genus $g\geqslant 2$, then $\# C(\mathbb{Q})\leqslant B_g$?
\end{question}

This is not even known for genus $g=2$. 

\begin{example}
Let $C:y^2=x^3+875 x$. Note that $C(\mathbb{Q})$ has the obvious point 
$(0,0)$, and one can do a bit of work to show that this is the only point. 
\end{example}

\begin{example}
Let $C:y^2=x^3+877 x$. Then $C(\mathbb{Q})$ once again contains $(0,0)$. A 
computer search showed that $(0,0)$ is the only point on $C$ of height 
$\leqslant 1000000$. For now, the \emph{height} of a solution is just the 
largest absolute value of the numerator / denominator of a solution written 
in reduced fractions. But there should be more solutions! Let $E$ be the 
projective curve over $\mathbb{Q}$ obtained by adjoining a point $O$ to $C$. 
As a Riemann surface, $E(\mathbb{C})$ is just a torus. 

We can give $E/\mathbb{Q}$ the structure of an abelian variety. That is, 
we can give $E$ the structure of a commutative algebraic group, i.e. the 
operation is given by rational functions, with $O$ being the identity. 

Later, we will think of $E\hookrightarrow \operatorname{Jac}(E)$, where 
$\operatorname{Jac}(E)$ is the Jacobian of $E$. This embedding is determined 
by a single point $O\in E$. 

In particular, $E(\mathbb{Q})$ is an abelian group with identity $O$. It is 
a theorem of Mordell that $E(\mathbb{Q})$ is in fact finitely generated. We 
know the structure of such groups: $E(\mathbb{Q})\simeq A\times \mathbb{Z}^r$, 
where $A$ is finite, and $r=\operatorname{rk} E$ is the \emph{rank} of $E$. 

In general, $A$ is computable. In our case, $A=\mathbb{Z}/2$. The hard 
computational problem is: ``what is $r$''? The Birch and Swinnerton-Dyer 
conjecture says that $r$ agrees with the order of vanishing $r'$ of a certain 
holomorphic function $L(E,s)$ at $s=1$. Sometimes, $r'$ can be computed.  
In our example, a computation shows that $r'=1$, so we expect 
$E(\mathbb{Q})\simeq \mathbb{Z}/2\times\mathbb{Z}$. In particular, 
$E(\mathbb{Q})$ should be infinite. One can show (using other methods) that 
$E(\mathbb{Q})=\langle (0,0),(x_0,y_0)\rangle$, where 
\[
  x_0 = \frac{37 5494 5281 2716 2193 1055 0406 9942 0927 9234 6201}{6215 9877 7687 1505 4254 6322 0780 6972 3804 4100}
\]
For details, see \cite{br84}. One method to construct such large solutions for 
a rank one elliptic curve uses Heegner points. 
\end{example}

In general, we will take a curve $C/\mathbb{Q}$, consider its jacobian $J$, 
and study $J(\mathbb{Q})$. This will be a group, and its structure heavily 
influences $C(\mathbb{Q})$. The ``average rank of an elliptic curve'' is not 
known, nor is there a general consensus on what is should be. Some expect it 
the rank of a random curve to be $0$ or $1$, both with probability 
$\frac 1 2$. Others suppose that elliptic curves over $\mathbb{Q}$ have rank 
$2$ with nonzero probability as well. It was proven recently (see 
\cite[\S 1]{bh10}) that 
\[
  \limsup_{B\to\infty} \frac{1}{4 B^2} \sum_{\substack{|a|,|b|\leqslant B \\ 4 a^3+27 b^2\ne 0}} \operatorname{rk}(E_{a,b}) \leqslant\frac 7 6
\]
where $E_{a,b}$ is the elliptic curve defined by $y^2=x^3 +a x+b$.

\begin{question}
Is $\operatorname{rk} E(\mathbb{Q})$ bounded for all elliptic curves 
$E/\mathbb{Q}$?
\end{question}

It is known that there are curves with rank at least $28$, but their exact 
ranks are not known \cite{du}. The largest known rank is $19$. 
%It is generally thought that the answer is no, and that the NSA has examples of curves with 
%much larger ranks than publicly known. 

In general, the assumption that $f(x,y)$ is absolutely irreducible is not a 
serious one. For example, if $f=y^2-x^2$, then we can factor $f$ as 
$(x+y)(x-y)$, and then treat the solutions to $x+y=0$ and $x-y=0$ separately. 
Another example is $f=x^2+y^2$, which only factors over $\mathbb{Q}(i)$ as 
$(y+i x)(y-i x)$, and we the rational points lie in the intersection of the 
two components over $\mathbb{Q}(i)$. Also, the assumption that 
$f\in \mathbb{Q}[x,y]$ is not serious. One can prove that every curve 
$C/\mathbb{Q}$ embeds into $\mathbb{P}^2_\mathbb{Q}$, hence is birational to 
a plane curve. 

Let $C/\mathbb{Q}$ be a curve. Then 
$C(\mathbb{C})\setminus \{\text{singular points}\}$ is a compact Riemann 
surface with points removed, i.e. it is a torus with $g$ handles with finitely 
many points removed. Call this $g$ the \emph{genus} of $C$. 

\begin{theorem}[Faltings, conjectured by Mordell]
If $C$ is a curve over $\mathbb{Q}$ with $g\geqslant 2$, then $C(\mathbb{Q})$ 
is finite. 
\end{theorem}

For curves of genus $g<2$, $\# C(\mathbb{Q})$ can be infinite. In Faltings' 
theorem, $\mathbb{Q}$ can be replace by any field finitely generated over 
$\mathbb{Q}$. 

Now let $C$ be a smooth projective curve of genus $g$ over $\mathbb{F}_p$. 
We are interested in $|C(\mathbb{F}_{p^n})|$, which is obviously computable 
for each $n$. Let 
\[
  Z(C,t) = \exp\left( \sum_{n>0} \# C(\mathbb{F}_{p^n}) \frac{t^n}{n} \right) \in \mathbb{Q}\llbracket t\rrbracket 
\]

\begin{theorem}[Weil]
If $C/\mathbb{F}_p$ is a smooth projective curve of genus $g$, then 
\[
  Z(C,t) = \frac{P(C,t)}{(1-t)(1-p t)}
\]
where $P(C,t)\in \mathbb{Z}[t]$ has degree $2 g$. Moreover, if 
$P(C,t) = \prod_{i=1}^{2 g} (1-\alpha_i t)$, then for each $i$, one has 
$|\alpha_i|=p^{1/2}$. 
\end{theorem}

The second statement in the theorem is called the \emph{Riemann hypothesis} 
for $C$. It can be used to obtain explicit bounds on the size of 
$C(\mathbb{F}_{p^n})$ as $n\to\infty$. For example, we can compute 
\begin{align*}
  \sum_{n>0} \# C(\mathbb{F}_{p^n}) \frac{t^n}{n} 
    &= \log Z(C,t) \\
    &= -\log(1-t) - \log(1-p t) + \sum_i \log(1-\alpha_t t) \\
    &= \sum_{n>0} \left(p^n+1-\sum_{i=1}^{2 g} \alpha_i^n\right) \frac{t^n}{n} \text{.}
\end{align*}
Therefore, $\# C(\mathbb{F}_{p^n}) = p^n+1-\sum_{i=1}^{2 g} \alpha_i^n$. It 
follows easily that $|\# C(\mathbb{F}_{p^n})-(p^n+1)| \leqslant 2 g p^{n/2}$. 
This is equivalent to saying 
\[
  p^n-2 g p^{n/2}+1 \leqslant \# C(\mathbb{F}_{p^n}) \leqslant p^n + 2 g p^{n/2} + 1
\]
In particular, setting $n = g = 1$, we obtain 
$\# C(\mathbb{F}_p) \geqslant (p^{1/2}-1)^2>0$, hence 
$C(\mathbb{F}_p)\ne\varnothing$. 










% notes taken on 09-03-2013

\section{Some analysis: jacobians over \texorpdfstring{$\mathbb{C}$}{C}}

In this section, we will mostly work over $\mathbb{C}$ and treat curves as 
Riemann surfaces. First, some more general definitions:

\begin{definition}
A \emph{variety} over a field $k$ is a separated scheme of finite type over 
$\spec(k)$. We call a variety $X/k$ \emph{nice} if it is smooth, projective, 
and geometrically integral.
\end{definition}

Recall that $X/k$ is \emph{geometrically integral} if 
$X_{\bar k}=X\times_k \spec(\bar k)$ is dimension $1$. A \emph{curve} is a 
variety of dimension one. If we are interested in $C(k)$ for general curves, 
it is sufficient to consider nice curves. If $X$ is a variety, we can consider 
its functor of points $h_X:\mathsf{Alg}_k\to \mathsf{Set}$ which assigns to a 
$k$-algebra $A$ the set $X(A)$ of ``$A$-valued points.'' This determines a 
functor $h_X:\mathsf{Sch}_k^\circ\to\mathsf{Set}$ which is defined by 
$h_X(Y)=\hom_k(Y,X)$. 

Earlier, for $f\in \mathbb{Q}[x,y]$ and $C$ defined by the equation $f=0$, we 
defined $C(A)=\{(a,b)\in A^2:f(a,b)=0\}$ for any $\mathbb{Q}$-algebra $A$. 
Note that 
\begin{align*}
  C(A) &= \{(a,b)\in A^2:f(a,b) = 0\} \\
    &= \hom_{\mathsf{Alg}_\mathbb{Q}}(\mathbb{Q}[x,y]/f,A) \\
    &= \hom_{\mathsf{Sch}_k}\left(\spec A,\spec(\mathbb{Q}[x,y]/f)\right) \\
\end{align*}
so this agrees with our general definition. 

\begin{remark}
The functor $h_X$ determines $X$, up to isomorphism. (This is just the Yoneda 
lemma).
\end{remark}

For the rest of this section, let $C$ be a nice curve over $\mathbb{C}$. Set 
$X=C(\mathbb{C})$; this is a compact connected Riemann surface. So 
topologically, $X$ is a many-handled torus. Let $\Lambda=\h_1(X,\mathbb{Z})$, 
the first singular homology group of $X$, which can be identified with 
$\pi_1(X)^\text{ab}$. So elements of $\Lambda$ are equivalence classes 
$[\gamma]$ for $\gamma\in \pi_1(X)$. 

\begin{remark}
The notation $\h_\bullet$, $\pi_1$\ldots will always denote the 
\emph{topological} homology, fundamental group\ldots. For, e.g. 
etale cohomology, we will write $\h_\text{et}^\bullet(C,\mathbb{Q}_\ell)$, or 
$\h_\text{crys}^\bullet(X/\mathbb{Z}_p)$ for crystalline cohomology.
\end{remark}

It is a theorem of algebraic topology that $\Lambda\simeq\mathbb{Z}^{2 g}$; we 
will call $g$ the \emph{genus} of $X$ (and also of $C$). Let $K$ be the field 
of meromorphic functions on $X$. In other words, elements of $K$ are of the 
form $f=g/h$ where $g$ and $h$ are holomorphic functions on $X$ and $g\ne 0$. 
(This forces $g\ne 0$ except on a finite subset of $X$.) At any point 
$x\in X$, we can write $f$ in local coordinates as $z^n(a_0+a_1 z+\cdots)$ 
where $a_0\ne 0$ and $n\in\mathbb{Z}$. We call $n=\ord_x(f)$ the \emph{order 
of vanishing} of $f$ at $x$. 
 
\begin{remark}
We can identify $K$ with the function field of $C$, i.e. the set of rational 
maps $C\to\mathbb{A}^1$. Certainly rational maps yield meromorphic functions, 
and it is a basic theorem of Riemann surface theory that meromorphic functions 
are in fact algebraic. Moreover, if $C$ is nice, then $C$ can be recovered 
from $K$. To do this, pick some $x\in K\setminus \mathbb{C}$. If we let $A$ be 
the integral closure of $\mathbb{C}[x]$ in $K$, then $\spec(A)$ will be a 
smooth affine curve. Pick some embedding of $\spec(A)$ into projective space; 
the closure of its image will be a projective curve $C'$ (possibly with 
singularities) with function field $K$. We can resolve the singularities of 
$C'$ to obtain a smooth projective curve $C''$ with function field $K$. 
Clearly $C''$ and $C$ are birational -- in fact this forces them to be 
isomorphic. Indeed, let $f:C''\to C$ be a birational map. By 
\cite[thm 3.1]{mi}, $f$ and $f^{-1}$ are in fact regular, hence $f$ is an 
isomorphism of varieties. 
\end{remark}

One might ask whether the singular homology $\h_1(X,\mathbb{Z})$ can be 
defined ``algebraically.'' Essentially, the answer is no -- that is, there is 
no known algebraic definition for $\h_1(X,\mathbb{Z})$ that gives the 
``right'' answers. On the other hand, $\h_1(X,\mathbb{Q})$ is naturally 
isomorphic to the dual of the algebraic de Rham cohomology 
$\h_\text{dR}^1(X/\mathbb{Q})$, and $\h_1(X,\mathbb{Z}_\ell)$ is naturally 
isomorphic to the dual of the $\ell$-adic cohomology 
$\h_\text{et}^1(X,\mathbb{Z}_\ell)$. Both of these isomorphisms are hard 
theorems -- the first follows from work of Grothendieck \cite{gr66}, the 
second from \cite[I 4.6.3]{de77}. 

With $C$ as before, let $V=\Omega^1(X)=\h^0(X,\Omega^1)=\h^1_\text{dR}(X)$. 
This is a complex vector space of dimension $g$. (Here, we mean analytic de 
Rham cohomology -- the algebraic de Rham cohomology would be denoted 
$\h_\text{dR}^1(C/\mathbb{C})$.) This gives us an algebraic definition of $g$. 
We can consider $\Omega^1$ as the sheaf of (algebraic) differentials, and 
$g=\dim_\mathbb{C}\h^0(X,\Omega^1)$. There is a pairing 
$\h_1(X,\mathbb{Z})\otimes\h_\text{dR}^1(X)\to\mathbb{C}$, defined by 
\[
  [\gamma]\otimes \omega \mapsto \int_\gamma \omega
\]
This pairing is $\mathbb{C}$-linear in the second component, and is in fact 
non-degenerate. It gives us a map $\Phi:\Lambda\to V^*$. 

\begin{definition}[analytic]
The \emph{Jacobian} of $X$ is 
$\jac X=V^*/\Phi(X) = \h_\textnormal{dR}^1(X)^*/\h_1(X,\mathbb{Z})$. 
\end{definition}

Is is known that $\Phi(\Lambda)$ is a lattice in $V^*$, i.e. it is discrete 
and $V^*/\Phi(\Lambda)$ is compact. There is the \emph{Abel-Jacobi map} from 
$X$ to $\jac X$ defined as follows. Fix $x_0\in X$; we send $x\in X$ to 
$\omega\mapsto \int_{[x_0,x]} \omega$, where $[x_0,x]$ denotes some path from 
$x_0$ to $x$. A different choice of $[x_0,x]$ will differ by a closed loop, 
i.e. an element of $\h_1(X,\mathbb{Z})$. So $X\to\jac X$ is well-defined. Note 
that $\jac X$ is a complex Lie group. 

\begin{remark}
After choosing a basis for $V^*$, we have $\jac X\simeq \mathbb{C}^g/L$, where 
$L\simeq \mathbb{Z}^{2 g}$. As a real Lie group, $\jac X$ is diffeomorphic to 
$(S^1)^{2 g}$. 
\end{remark}

We care about $\jac X$ because, despite its analytic definition, it is in fact 
a projective variety.

\begin{theorem}
For $X$ a compact Riemann surface, $\jac X$ is algebraic, i.e. there exists a 
variety $J$ defined over $\mathbb{C}$ such that $J(\mathbb{C})\simeq\jac X$ as 
complex manifolds. Moreover, the group operation on $\jac X$ is algebraic, 
i.e. there is a morphism $m:J\times_\mathbb{C} J\to J$ such that 
$J(\mathbb{C})\times J(\mathbb{C})\to J(\mathbb{C})$ corresponds to the 
addition law on $\jac X$. 
\end{theorem}
\begin{proof}
look up in \cite{mu08}
\end{proof}

Let $\Div X$ be the free abelian group generated by the points of $X$. There 
is a map $\deg:\Div X\to \mathbb{Z}$, defined by 
$\sum n_x\cdot x\mapsto \sum n_x$. We define $\Div^\circ X$ by the short exact 
sequence 
\[
  0 \to \Div^\circ X \to \Div X \to \mathbb{Z} \to 0
\]
There is also a map $\operatorname{div}:K^\times\to \Div X$, where 
$\operatorname{div}(f) = \sum_x \ord_x(f)\cdot x$. It is a basic fact that 
$\deg(\operatorname{div}(f)) = 0$, so we can define the \emph{Picard group} 
of $X$ to be $\pic X = \Div X/\operatorname{div}(K^\times)$ and 
$\pic^\circ X = \Div^\circ (X)/\operatorname{div}(K^\times)$. 

\begin{remark}
Let $\mathscr{M}$ be the sheaf of meromorphic functions on $X$. One can 
prove that $\Div(X)=\h^0(X,\mathscr{M}^\times/\mathscr{O}^\times)$ and 
$\pic(X)=\h^1(X,\mathscr{O}^\times)$. 
\end{remark}

\begin{example}
One can prove that $\pic^\circ(\mathbb{P}^1) = 0$.
\end{example}

The Abel-Jacobi map $\mu:X\to \jac X$ extends to a map 
$\mu:\Div^\circ X\to \jac X$. 

\begin{theorem}[Jacobi]
The map $\mu:\Div^\circ X\to\jac X$ is surjective.
\end{theorem}

\begin{theorem}[Abel]
The kernel of $\mu:\Div^\circ X\to \jac X$ is $\operatorname{div}(K^\times)$. 
\end{theorem}

It follows that $\mu$ induces an isomorphism $\mu:\pic^\circ X\to \jac X$. 
Note that $\pic^\circ X$ parameterizes invertible sheaves (line bundles) on 
$X$ of degree zero. 

Note that in general, $\mathbb{C}^g/L$ for some lattice $L$ need not be 
algebraic if $g>1$. In the future, we'll try to define $\jac C$ for a curve 
$C$ over an any field. The variety $\jac C$ will be a nice variety, i.e. 
smooth, projective and geometrically integral. We will use this to give an 
algebraic definition of $\h_1(X,\mathbb{Z}/n)$. 










%\begin{thebibliography}{9}
%  \bibitem{bh10} M. Barghava, A. Shankar, \emph{Ternary cubic forms having bounded invariants, and the existence
%  of a positive proportion of elliptic curves having rank $0$}, preprint, \texttt{arXiv:1007.0052}.
  %\bibitem{br84} A. Bremner, J. Cassels, \emph{On the equation $Y^2=X(X^2+p)$}, Math. Comp. 42 (1984), no.165, 257-264. 
  %\bibitem{de77} P. Deligne, \emph{Cohomologie Etale (SGA 4$\frac 1 2$)}, Springer Lecture notes 569, 1977. 
  %\bibitem{du} A. Dujella, \emph{History of elliptic curves rank records}, \url{http://web.math.pmf.unizg.hr/~duje/tors/rankhist.html}. 
  %\bibitem{gr66}. A. Grothendieck, \emph{On the de Rham cohomologie of algebraic varieties}, Pub. Math. I.H.E.S. \textbf{29} (1966), 95-103. 
 % \bibitem{ma} ``what is the maximum number of rational points of a curve of genus 2 over the rationals?'', question on mathoverflow, \url{http://mathoverflow.net/questions/103327}. 
 % \bibitem{mi} J.\ S. Milne, \emph{Abelian varieties}, available online at \url{http://www.jmilne.org/math/CourseNotes/AV.pdf}.
 % \bibitem{mu08} D. Mumford, \emph{Abelian varieties}, Tata inst. fund. res., 2008. 
 % \bibitem{st09} M. Stoll, \emph{On the average number of rational points on curves of genus 2}, preprint, \texttt{arXiv:0902.4165}. 
%\end{thebibliography}

\bibliographystyle{amsplain}
\bibliography{7390-sources}




\end{document}
